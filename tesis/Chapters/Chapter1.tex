\chapter{Introducción al Trabajo de Tesis}

El presente trabajo de tesis se encuentra enfocado en el contexto del diseño de hardware digital. El
mismo fue motivado por el creciente avance en las tecnologías utilizadas en los sistemas de
comunicación en general y en los sistemas \textit{Software Defined
Radio}\footnote{\label{SDR}Software Defined Radio (SDR por sus siglas en inglés): Sistema de
comunicaciones donde los componentes típicamente implementados en hardware (mezcladores, filtros, amplificadores, moduladores / demoduladores, detectores, etc) son
implementados en software.} en particular.
Los sistemas \textit{SDR} permiten implementar sistemas de comunicación mediante software o hardware digital reconfigurable, otorgándoles una gran flexibilidad.\\

La creciente demanda de velocidad en las telecomunicaciones lleva a la implementación de sistemas de
transmisión cada vez más veloces. Uno de los sistemas de transmisión de datos más difundidos es el
sistema \textit{OFDM}, en el cual se utilizan múltiples portadora en las que se modulan los
datos a trasmitir. Una forma práctica y eficiente de implementar las modulación y demodulación
multiportadora requerida por este sistema es mediante el uso de la Transformada Discreta de
Fourier, aprovechando los algoritmos de alta eficiencia disponibles para su implementación.\\

Teniendo en cuenta la complejidad de los sistemas de transmisión \textit{OFDM} y la necesidad de
poder implementarlos en forma eficiente tanto en consumo como en espacio y recursos utilizados, el
presente trabajo de tesis se basa en el desarrollo de un sistema de modulación y demodulación para
transmisiones \textit{OFDM} con muy baja ocupación espacial y de recursos, de manera que pueda ser
integrado fácilmente en sistemas de comunicación reducidos.

\section{Objetivo}
El trabajo de tesis tiene como objetivo diseñar e implementar en hardware digital distintas
arquitecturas capaces de realizar el cálculo de la Transformada Discreta de Fourier (\textit{DFT}
por sus siglas en inglés), teniendo como requerimientos sobre las arquitecturas desarrolladas que
realicen el cálculo en forma eficiente y sean de tamaño reducido. El diseño e implementación será
realizado mediante el lenguaje de modelado de hardware Verilog, orientado a la implementación en
ASIC, mientras que la validación se realizará en un dispositivo FPGA\footnote{\label{FPGA}Field Programmable
Gate Array:
dispositivo semiconductor que contiene bloques de lógica cuya interconexión y funcionalidad
puede ser configurada `in situ' mediante un lenguaje de descripción
especializado.}.\\

El desarrollo de la tesis se divide en una parte teórica y una experimental. Se analizarán distintos
desarrollos algorítmicos para el cálculo de la Transformada Discreta de Fourier, se seleccionarán
los que se consideren más aptos de acuerdo a los requerimientos para las arquitecturas a desarrollar
y se realizará un análisis para el diseño de IP cores que implementen los algoritmos seleccionados.
Una vez finalizado el diseño e implementación se realizarán una serie de pruebas y ensayos de
verificación de los IP cores generados, para finalizar con la validación de los
mismos a través de la síntesis en FPGA utilizando un kit de desarrollo. Se analizarán los datos
obtenidos de cada simulación y prueba para la determinación de las características de los diseños
implementados, tales como el error y la distorsión total armónica.


\section{Alcance}

Como resultados a obtener de la presente tesis se tienen los siguientes:

\begin{itemize}
    \item IP Cores codificados en el lenguaje Verilog de arquitecturas de
    cálculo de FFT de tamaño reducido.
    \item Estudio del comportamiento numérico de las arquitecturas implementadas (ruido, distorsión
    armónica, etc.)
    \item Análisis comparativos de procesamiento entre las arquitecturas
    desarrolladas y desarrollos de terceros.
    \item Proposición de trabajos futuros y/o mejoras.
\end{itemize}

\section{Organización del Trabajo}
En esta sección se describe la organización de la presente tesis. Con el objetivo de que la misma sea
autocontenida, los primeros capítulos se ocupan de presentar las bases o conocimientos necesarios para comprender la
totalidad del trabajo.

El desarrollo de la tesis se organiza de la siguiente forma:

\begin{itemize}
\item En el capítulo 2 se hará referencia a los conceptos requeridos para comprender el desarrollo
de la Transformada Discreta de Fourier y su utilización en sistemas \textit{OFDM}. Se desarrollarán
los conceptos teóricos matematicos que describen la Transformada de Fourier y se realizará una breve
descripción de los sistemas de comunicación \textit{OFDM} para brindar el marco teórico en el que se
encuadra el trabajo de tesis.
\item En el capítulo 3 se describirán los diferentes algoritmos para el cálculo de la DFT. Se
elegirán aquellos que se consideren aptos para la implementación en base a un determinado criterio.
Se analizarán también las diferentes alternativas que puedan surgir para la implementación de las
distintas partes componentes de los algoritmos seleccionados y se elegirán las más indicadas de
acuerdo a determinados criterios que serán expuestos cuando se requieran.
\item En el capítulo 4 se describirá la implementación en Verilog de los algoritmos seleccionados.
Se generará un IP core en RTL para cada algoritmo implementado. Dichos RTL deberán cumplir con
ciertas condiciones de portabilidad y legibilidad de código para que los mismos sean efectivamente
IP Cores.
\item En el capítulo 5 se expondrán los bancos de prueba de simulación y los resultados obtenidos de
las mismas. También se realizará la implementación sobre dispositivos FPGA midiendo los recursos
utilizados y comparándolo con implementaciones de terceros para verificar el cumplimiento de los
requerimientos de diseño de las arquitecturas.
\item En el capítulo 6 se extraerán las conclusiones pertinentes sobre los resultados obtenidos y
se propondrán futuras mejoras de las arquitecturas a partir del análisis realizado.

\end{itemize}
