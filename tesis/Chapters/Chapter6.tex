\chapter{Conclusiones}

\section{Conclusiones Generales}

El alcance propuesto para el presente trabajo se basó en los siguientes puntos:

\begin{itemize}
    \item IP Cores codificados en el lenguaje Verilog de arquitecturas de
    cálculo de FFT de tamaño reducido.
    \item Estudio del comportamiento numérico de las arquitecturas implementadas (ruido, distorsión
    armónica, etc.)
    \item Análisis comparativos de procesamiento entre las arquitecturas
    desarrolladas y desarrollos de terceros.
    \item Proposición de trabajos futuros y/o mejoras.
\end{itemize}

Se logró implementar satisfactoriamente dos arquitecturas diferentes basadas en el algoritmo Radix
para el cóomputo de la FFT, ofreciendo además la posibilidad de optar por dos métodos diferentes de
multiplicación por los \textit{twiddle factors}, siendo más eficiente el multiplicador complejo en
FPGAs que posean multiplicadores dedicados.

Dichas implementaciones se resumen en 20 códigos fuente para cada una, de los cuales 17 son
compartidos entre ambas, que contienen la lógica y descripción del hardware en lenguaje Verilog.
Adicionalmente se desarrollaron diferentes herramientas que permiten ensayar las arquitecturas en
diferentes condiciones de forma automática.

Como parte de los ensayos de las arquitecturas se obtuvieron diferentes métricas que permiten
caracterizarlas. Por un lado se presentó el resumen de recursos utilizados para la síntesis de las
arquitecturas posibles para una FPGA y se lo comparó con los recursos utilzados por arquitecturas
desarrolladas por terceros, incluyendo una comercial. De esta comparación se concluye que se cumple
con el requerimiento de mínima área de chip necesaria y la economía de recursos utilizados,
resultando más eficiente espacialmente que las soluciones de terceros. De este modo, resulta en una
opción ventajoda para implementar en sistemas SDR con recursos limitados.

La distorsión armónica total medida se encuentra en el orden de desarrollos de terceros ampliamente
utilizados, lo que indica que las arquitecturas desarrollas en el presente trabajo de tesis son
aptas para seer utilizadas en sistemas de comunicación con gran confiabilidad.

El error relativo medido, utilizando como parámetro de medidición un sistema con precisión de punto
flotante de 64 bits, es comparable con implementaciones de terceros utilizadas comercialmente en
procesamiento de señales y sistemas de comunicación, por lo que las arquitecturas desarrolladas son
aptas para su utilización en dichos sistemas.

De este modo se concluye en que se obtuvieron dos arquitecturas, con sus variantes, de baja
utilización de recursos, aptas para ser utilizadas en sistemas reales de comunicación y
procesamiento de señales, cumpliendo con los objetivos planteados al comienzo del trabajo de
tesis.

\section{Trabajos Futuros}

Este trabajo se presentó la primer etapa en el desarrollo de las arquitecturas, quedando
a futuro varios aspectos a mejorar y posibilidades de optimización del diseño.
Como potenciales trabajos a futuro en el contexto de la presente tesis, se destacan los siguientes:

\begin{enumerate}
	\item Estudiar posibles implementaciones de algoritmos de \textit{dithering} para reducir el ruido
	generado en las arquitecturas.
	\item Modificar el módulo de rotación Cordic agregando un pipeline que permita aumentar la
	velocidad de \textit{clock} de las arquitecturas, sin agregar ciclos de \textit{clock} extra al
	cómputo total de la FFT.
	\item Modificar el multiplicador complejo agrandando el tamaño de palabra de los factores de
	multiplicación de los \textit{twiddle factors} y/o optimizarlo para la utilización de los bloques
	de procesamiento digital de los dispositivos FPGA.
	\item Modificar el hardware para que el tamaño de palabra del ángulo de rotación sea independiente
	del tamaño de palabra de las señales.
	\item Estudiar la posibilidad de modificar las arquitecturas de forma de poder modificar la
	cantidad de puntos de la FFT en forma dinámica.
\end{enumerate}
